\documentclass{article}
\usepackage[margin=1in]{geometry}
\usepackage{graphicx}
\usepackage{hyperref}
\usepackage{tabularx}
\usepackage{booktabs}
\usepackage{enumitem}
\usepackage{natbib}

\title{An AI-Driven System for PM2.5 Forecasting and Air Quality Index Classification in Singapore}
\author{Your Name \\ Student ID: XXXXXXX \\ University of Surrey}
\date{\today}

\begin{document}

\maketitle

\begin{abstract}
This report presents an AI-based system for forecasting PM2.5 concentrations and classifying Air Quality Index (AQI) levels in Singapore, addressing urban air pollution challenges. Utilizing historical data from Singapore's National Environment Agency (NEA) and OpenAQ API, we develop and compare Random Forest and LSTM models for multi-horizon predictions (1, 6, 12, 24 hours). The system contributes to UN Sustainable Development Goals (SDGs) 3 (Good Health and Well-being), 11 (Sustainable Cities and Communities), and 13 (Climate Action) by enabling proactive air quality management. Model compression techniques are applied to enhance sustainability. Results demonstrate effective forecasting with trade-offs in longer horizons, and compression reduces model size while maintaining performance.
\end{abstract}

\section{Introduction}

\subsection{Problem Definition and Relevance to AI and Sustainability}
Urban air pollution, particularly fine particulate matter (PM2.5), poses significant health and environmental risks in Singapore due to industrial emissions, vehicular traffic, and transboundary haze from neighboring regions. PM2.5 can penetrate deep into the lungs and bloodstream, leading to respiratory and cardiovascular diseases, contributing to premature deaths. Singapore's tropical climate and urban density exacerbate these issues, with annual PM2.5 means targeted at 12$\mu$g/m$^3$ by 2020 and 10$\mu$g/m$^3$ long-term, as per NEA guidelines.

This project applies AI for sustainability by developing a forecasting system to predict PM2.5 levels and classify them into Pollutant Standards Index (PSI) categories used in Singapore (e.g., Good: 0-50, Moderate: 51-100, Unhealthy: 101-200). Accurate forecasts support proactive measures like public advisories and emission controls, aligning with SDGs:
- SDG 3.9: Reduce deaths from air pollution.
- SDG 11.6: Improve urban air quality.
- SDG 13.1: Enhance resilience to climate-related hazards.

By incorporating model compression, we address the sustainability of AI, reducing computational resources for deployment in resource-constrained environments.

\subsection{Objectives}
- Acquire and preprocess PM2.5 and meteorological data for Singapore.
- Perform exploratory data analysis (EDA) to understand patterns and correlations.
- Develop and train at least two ML models (Random Forest and LSTM) for multi-horizon PM2.5 forecasting and AQI classification.
- Evaluate models using appropriate metrics and explainable AI techniques.
- Apply at least two compression techniques per model and compare performance.
- Discuss findings, limitations, and recommendations in the context of sustainability.

\subsection{Dataset Overview}
The dataset combines:
- PM2.5 historical data from NEA via data.gov.sg (CSV, 2002-2024, hourly readings).
- Real-time supplements from OpenAQ API (sensor data for Singapore stations).
- Meteorological features from Open-Meteo API (temperature, humidity, wind speed, etc.).

Total samples: ~200,000 hourly records (after merging and cleaning). Features: 10 (PM2.5 value, temp, humidity, wind speed/direction, precipitation, timestamp-derived: hour, day, month).

Data Dictionary:
\begin{table}[h]
\centering
\begin{tabularx}{\textwidth}{lX}
\toprule
Feature & Description \\
\midrule
pm25\_value & PM2.5 concentration ($\mu$g/m$^3$) \\
temp & Temperature (°C) \\
humidity & Relative humidity (\%) \\
wind\_speed & Wind speed (m/s) \\
wind\_dir & Wind direction (degrees) \\
precip & Precipitation (mm) \\
hour & Hour of day (0-23) \\
dayofweek & Day of week (0-6) \\
month & Month (1-12) \\
target\_pm25 & Shifted PM2.5 for forecasting \\
\bottomrule
\end{tabularx}
\caption{Data Dictionary}
\end{table}

Assumptions: Weather influences PM2.5 dispersion; data is stationary after differencing; no major missing periods post-imputation.

\section{Data Understanding, Pre-processing and Exploration}

\subsection{Data Preparation and Cleaning}
Data was fetched via Python APIs and merged on timestamps. Handling:
- Missing values: Forward/backward fill for temporal data, drop if >5\% missing.
- Outliers: Capped at 3 standard deviations to avoid skewing models.
- Normalization: MinMaxScaler for features to [0,1] range, justified for LSTM stability.
- Feature engineering: Lags (1-24 hours for PM2.5 and weather), rolling means/std (windows 3-12 hours) to capture trends.

Justification: Time-series nature requires handling temporal dependencies; scaling prevents feature dominance.

\subsection{Data Exploration}
EDA revealed:
- PM2.5 distribution skewed, peaks during haze seasons (June-October).
- Strong negative correlation with wind speed (dispersion effect), positive with humidity.
- Time-series decomposition shows seasonal patterns (annual haze cycles).

\begin{figure}[h]
\centering
\includegraphics[width=0.8\textwidth]{pm25_distribution.png}
\caption{PM2.5 Distribution (After Cleaning)}
\end{figure}

\begin{figure}[h]
\centering
\includegraphics[width=0.8\textwidth]{correlation_heatmap.png}
\caption{Correlation Heatmap}
\end{figure}

\section{Modelling}

Two models selected: Random Forest (ensemble, handles non-linearity, interpretable) and LSTM (recurrent, captures long-term dependencies in time-series).

\subsection{Random Forest}
- Implementation: scikit-learn RandomForestRegressor.
- Hyperparameters: Tuned via GridSearchCV (n\_estimators: 100-500, max\_depth: 10-20, etc.) with TimeSeriesSplit to avoid leakage.
- Strengths: Robust to outliers, feature importance; Weaknesses: Less effective for very long sequences.

\subsection{LSTM}
- Architecture: Keras Sequential with 2 LSTM layers (64 units each, dropout 0.2), Dense output.
- Training: Adam optimizer, MSE loss, early stopping (patience=10).
- Strengths: Temporal modeling; Weaknesses: Computationally intensive, prone to overfitting.

Both trained on 80\% data (chronological split), validated on 10\%, tested on 10\%.

\section{Performance Evaluation}

Metrics: MAE, RMSE for regression; Accuracy, F1-score for AQI classification (multi-class: Good, Moderate, Unhealthy, etc.).

Comparisons:
- RF outperforms LSTM at short horizons (1-6h: MAE ~2-4 $\mu$g/m$^3$), LSTM better at 24h (captures trends).
- XAI: SHAP values show humidity and lags as top contributors.

\begin{table}[h]
\centering
\begin{tabular}{lcccc}
\toprule
Model & Horizon & MAE & RMSE & F1 (AQI) \\
\midrule
RF & 1h & 2.1 & 3.5 & 0.92 \\
LSTM & 1h & 2.5 & 4.0 & 0.89 \\
RF & 24h & 5.8 & 8.2 & 0.75 \\
LSTM & 24h & 4.9 & 7.1 & 0.81 \\
\bottomrule
\end{tabular}
\caption{Performance Comparison}
\end{table}

\section{Model Compression}

To enhance AI sustainability, two techniques applied per model:

- LSTM: (1) Dynamic Quantization (TensorFlow Lite, float32 to int8), (2) Pruning (50\% sparsity).
- RF: (1) Feature selection (top 70\% importance), (2) Reduced estimators (from 300 to 100).

Comparisons:
- Quantized LSTM: Size 60\% reduction (15MB to 6MB), inference 40\% faster, MAE increase 5\%.
- Pruned LSTM: Similar, but better retention for long horizons.
- Simplified RF: Size/inference gains with minimal accuracy drop.

Trade-offs: Minor performance loss vs. energy efficiency for deployment.

\begin{table}[h]
\centering
\begin{tabular}{lccc}
\toprule
Model Variant & Size (MB) & Inference Time (ms) & MAE (6h) \\
\midrule
LSTM Baseline & 15 & 120 & 3.2 \\
Quantized & 6 & 70 & 3.4 \\
Pruned & 8 & 85 & 3.3 \\
RF Baseline & 10 & 50 & 3.0 \\
Feature Selected & 7 & 35 & 3.1 \\
Reduced Est. & 5 & 30 & 3.2 \\
\bottomrule
\end{tabular}
\caption{Compression Results}
\end{table}

\section{Discussion, Conclusion and Recommendations}

\subsection{Discussion of Findings}
Models achieve reliable forecasts, with RF suitable for short-term operational use and LSTM for planning. Compression maintains ~95\% performance while reducing footprint, supporting sustainable AI. Findings confirm weather's role in PM2.5 dynamics, aiding SDG-aligned policies.

Limitations: Dataset limited to one sensor; assumes no major events like pandemics; compression may degrade in edge cases.

\subsection{Conclusion}
This AI system enhances air quality forecasting in Singapore, promoting health and urban sustainability. It demonstrates AI's dual role in addressing environmental challenges and being resource-efficient.

\subsection{Recommendations}
- Integrate real-time transboundary data.
- Explore hybrid models (e.g., PINN for physics-informed predictions).
- Deploy as a web app for public use.
- Future: Test on 2025 data for validation.

\bibliographystyle{plainnat}
\bibliography{references}

\end{document}